\chapter{定式化}\label{formulation}
乗合タクシー問題は、$G = (V, E)$の完全有向グラフ上で定義される.$V =\{\{0,2n+1\},P,D\}$を頂点集合とし, $A = \{(i,j) \mid i,j\in V ,i \neq j\}\}$を各頂点間の辺集合とする.
ここで,デポを$0$と$2n+1$で表し, $V$の部分集合$P =\{1,...,n \}$を乗車地点の集合, $V$の部分集合$D =\{n+1,...,2n \}$を降車地点の集合とする.
$i \in P$, $i+n \in D$に対して,$V$上の2点のペア$(i,i+n)$は乗車地点の頂点$i$から目的地に対応する頂点$i+n$への乗客輸送リクエスト(以下、単にリクエストと呼ぶ)とする.
それぞれの頂点$v_i \in V$には, 負荷$q_i$とサービス時間$s_i$が与えられる.

各リクエストは, $m$台の車両$k \in K = \{1,2,..m\}$で訪問される.
$\sigma$をルートの集合とし, $\sigma_k$を車両$k$の訪問する頂点の順列とすると, $\sigma = \{\sigma_1,...,\sigma_m\}$である. また, $\sigma_k(h)$は車両$k$が$h$番目に訪問する頂点をあらわす.
頂点$i,j \in V$間の距離を$c_{ij}$, 時間を$t_{ij}$とする.
車両の容量を$Q$とし, 車両$k$の$\sigma _{k(h)}$における容量を$q_{\sigma_k(h)}$とする.
$x_{ij}$を0-1変数とし, 頂点$i,j$を結ぶ辺がルートに含まれれば1,そうでなければ0とする.
各車両は1つのデポから出発しデポに帰る. ここで, $n_k$は車両$k$が訪問するデポ以外の頂点の数である

この問題に対して、目的関数として
\begin{enumerate}
 \item 車両運用コスト最小化,
 \item 顧客満足度最大化
\end{enumerate}
の2つを考える。車両運用コストとしては、様々なコストが考えられるが本研究では車両が走行するルートの総距離とする。ルートの総距離を$d(\sigma)$とすると,
\begin{align*}
d(\sigma) = \sum_ {k\in K} \sum_{h=0}^{n_k} c_ {\sigma_{k(h)},\sigma_ {k(h+1)} }\\
\end{align*}
と表せる.
各リクエストにおける乗車時刻に対するペナルティ関数を$g^+_i$, 降車時刻に対するペナルティ関数を$g^-_i$, 乗車時刻に対するペナルティ関数を$g_i$とする.
頂点$i$でのサービス開始時刻を$\tau_i$とし, Iをリクエスト集合として, $I_\sigma$をルート$\sigma$に属するリクエストの集合とする.
また, 利用者の不満度を$t(\sigma)$とすると,
\begin{align*}
t(\sigma) = \sum_ {i \in I_\sigma} (g^+_i(\tau_i)+g^-_i(\tau_{i+n})+g_i(\tau_{i+n}-\tau_i))
\end{align*}
と表せる.
本研究では、2つの目的関数の重み付き和の最小化を考える。そうすることで、少し顧客の不満度が上がっても良いのでルートの総距離を短くしたい場合や、顧客の不満度を最優先で考えたい場合など、様々な状況を考慮することができる。

それぞれの係数を定数$\alpha$と$\beta$としたとき、以下のように定式化できる:
\begin{align*}
  &\textrm{minimize}   &&
  \alpha d(\sigma)+ \beta t(\sigma),\tag{1}\\
  &\textrm{subject to} && \sum_{i \in V} x_{ij} = 1 \ \ \ \ \ \ \ \ \ \ \ \ \ \ \ \ \ \ \ (j \in  P \cup D), \tag{2}\\
  &                    && \sum_{j \in V} x_{ij} = 1 \ \ \ \ \ \ \ \ \ \ \ \ \ \ \ \ \ \ \ (i \in  P \cup D), \tag{3}\\
  &                    && \sum_{k \in K} n_k = 2n,\tag{4}\\
  &                    && \sigma_k(0) = \sigma_k(n_k+1) = 0 \ \ \ \ \ \ (k \in K),\tag{5}\\
  &                    && 0 < q_{\sigma_k(h)} < Q\ \ \ \
  (h \in \{0,...,n_k\},k \in K),\tag{6}\\
\end{align*}
\begin{align*}
  &                    && g_i(\tau_{i+n}-\tau_i) > s_i + t_{i,i+n}\ \ \ \ \ \ \ \ \ \ \ \ \ \  (i \in I),\tag{7}\\
  &                    && \tau_ {\sigma_k (h+1)} \geqq \tau_ {\sigma_k (h)} + s_{\sigma_k (h)} + t_ {\sigma_k (h),\sigma_k (h+1)}\\
  &                    &&  \ \ \ \ (h \in \{0,...,n_k\},k \in K).\tag{8}
\end{align*}
式$(2),(3)$は訪問点が1回ずつしか訪問されないことを, 式$(4)$は全てのリクエストが実行されることを表している. 式$(5)$は全ての車両がデポから出発してデポに帰ることを表している. 式$(6)$は車両の容量制約を表している. 式$(7),(8)$はリクエストの先行制約とサービス時刻に関する制約を表している.
