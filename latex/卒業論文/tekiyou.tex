\begin{center}
{\LARGE 時間枠及び乗車時間ペナルティ付き\\乗合タクシー問題に対する局所探索法}\\[0.5cm]
\end{center}
\hfill
{\large 051600141\qquad 竹田 陽}\\[0.5cm]
\begin{center}
{\Large \bf 概 要}\\
\end{center}

近年,乗合タクシーサービスは新しい移動手段として需要が増加してきている.また,老人や介護が必要な人を自宅などからヘルスケアセンターなどまで輸送するようなサービスも,高齢化に伴い需要が増加してきている.これらのサービスの特徴としては,利用者はリクエストとして出発地と到着地,それぞれに対して場所と時間枠を指定することができる.本研究では,これらのようなサービスにおいて利用者の満足度を考慮しつつサービスの実行にかかるコストを最小化することを考える.このような問題はDial-a-Ride問題(Dial-a-Ride problem, DARP)と呼ばれ,多くの研究がなされている.

多くの先行研究では,利用者の最大乗車時間とリクエストの乗車時間と降車時間に対しての時間枠をハード制約として与えている.
本研究では,最大乗車時間と時間枠をペナルティ関数で与えてソフト制約とする.こうすることで,時間枠より少しの遅延は許容できる場合など,様々なケースを柔軟に考慮することができるようになり,先行研究より汎用的な問題とすることができる.事前に全てのリクエストがわかっている問題を静的DARP,リクエストが全てはわかっておらず問題を解く過程でリクエストが次々と与えられる問題を動的DARPといい,本研究では静的DARPを考える.
