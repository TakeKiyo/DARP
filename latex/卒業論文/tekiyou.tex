\begin{center}
{\LARGE 時間枠及び乗車時間ペナルティ付き\\乗合タクシー問題に対する局所探索法}\\[0.5cm]
\end{center}
\hfill
{\large 051600141\qquad 竹田 陽}\\[0.5cm]
\begin{center}
{\Large \bf 概 要}\\
\end{center}


乗合タクシー問題(Diai-a-ride problem)とは,異なる移動需要を持つ顧客を同じ車両で同時に輸送する際に,効率の良い車両の割り当てとスケジューリングを考える問題である. 乗合タクシー問題では,顧客の乗降点と到着時刻に対する時間枠制約,乗車時間に対する制約が与えられたときに,車両の運用にかかるコストおよび顧客の不満度合いを最小にすることを目的とする.

乗合タクシー問題の制約としては,車両の容量資源の最大容量を超えて乗車することはできない(容量制約),車両が回るルートの大きさ(最大ルート距離),顧客が車両に乗っている時間の長さ(最大乗車時間)が与えられる.

本研究では,時間枠と乗車時間に対する制約を凸なペナルティ関数として与えて,その問題に対して局所探索を行った.
