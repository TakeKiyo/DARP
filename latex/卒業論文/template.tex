%#!ptex2pdf -l -u -ot '--synctex=1 --shell-escape' template
%\documentclass[uplatex]{jsarticle}
\documentclass[uplatex]{ujreport}
\usepackage{booktabs}
\usepackage{amsmath}
\usepackage{amssymb} 
\usepackage{psfrag}
\usepackage[T1]{fontenc }
\usepackage{pifont}
\usepackage{algorithm}
\usepackage{algorithmic}
\usepackage{amsmath,cases}
\usepackage{here}
\usepackage[dvipdfmx]{graphicx}
\usepackage{url}      %URLの表記に使う\urlコマンドに必要.
\usepackage{enumerate}%enumerate環境で項目を[Step 1.]のような形式に変更するのに利用.


\setlength{\topmargin}{-10mm}
\setlength{\textheight}{23cm}
\setlength{\oddsidemargin}{5mm}
\setlength{\evensidemargin}{5mm}
\setlength{\textwidth}{15cm}

\renewcommand{\tablename}{表}
\renewcommand{\figurename}{図}
%\newcommand{\bs}{\texttt{\symbol{'134}}}
%    \newcommand{\cmd}[1]{\texttt{\def\{{\char`\{}\def\}{\char`\}}\bs#1}}
\newtheorem{thm}{定理}[section]
\newtheorem{prf}{証明}[section]
\usepackage{latexsym}
\def\qed{\hfill $\Box$}

\usepackage[margin=3.25cm]{geometry}
\renewcommand{\bibname}{参考文献}

\begin{document}





%%%%%%%%%%%%%%%%%%%%%%%%%%%%%%%%%%%%%%%%%%%%%%%%%
%表紙
\begin{table}[b]
\begin{center}
{\huge 修\hspace{0.1cm} 士\hspace{0.1cm} 論\hspace{0.1cm} 文}\\[2.5cm]
{\huge 麻雀における最適戦略確立のための\\GRASP法を用いた近似アルゴリズムの検討}\\[6cm]
{\huge 351401074\qquad 澤井佑樹}\\[1cm]
{\huge 名古屋大学大学院情報科学研究科}\\[0.5cm]
{\huge 計算機数理科学専攻}\\[0.5cm]
{\huge 2016年1月}\\
\end{center}
\end{table} 
%%%%%%%%%%%%%%%%%%%%%%%%%%%%%%%%%%%%%%%%%%%%%%%%%


\thispagestyle{empty} 
\clearpage
\newpage
\pagenumbering{roman}
\setcounter{page}{1}


%%%%%%%%%%%%%%%%%%%%%%%%%%%%%%%%%%%%%%%%%%%%%%%%%
%摘要とabstract
\begin{center}
{\LARGE 時間枠及び乗車時間ペナルティ付き\\乗合タクシー問題に対する局所探索法}\\[0.5cm]
\end{center}
\hfill
{\large 051600141\qquad 竹田 陽}\\[0.5cm]
\begin{center}
{\Large \bf 概 要}\\
\end{center}


乗合タクシー問題(Diai-a-ride problem, DARP)とは, 異なる移動需要を持つ顧客を同じ車両で同時に輸送する際に, 効率の良い車両の割り当てとスケジューリングを考える問題である. 乗合タクシー問題では, 顧客の乗降点と到着時刻に対する時間枠制約, 乗車時間に対する制約が与えられたときに, 車両の運用にかかるコストおよび顧客の不満度合いを最小にすることを目的とする.

乗合タクシー問題では, 利用者を出発地から目的地まで輸送するサービスにおいて, リクエスト全ての点を訪問すること, 車両はデポから出発しデポに戻ること, リクエストのペアは必ず同じ車両が訪問すること, 出発地より後に到着点を訪問することの全てを満たすルートと車両の割り当てを考える問題である. 利用者は, 出発地と目的地の場所, またその地点に車両が到着してほしい時間枠を指定することができる.
また, 輸送の時間に対して, 最大乗車時間を指定することができる.
乗合タクシー問題に対しては, 多くの研究がなされている. 多くの先行研究では, 利用者の最大乗車時間とリクエストの乗車時刻と降車時刻に対しての時間枠をハード制約として与えている.

時間枠と乗車時間に対する制約を区分線形で凸のペナルティ関数として与えることで, 乗合タクシー問題の顧客の不満度合いをより柔軟に表現できるようにした. その問題を「時間枠及び乗車時間ペナルティ付き乗合タクシー問題」と定義する.

本研究では, 時間枠及び乗車時間ペナルティ付き乗合タクシー問題に対して, ルート内の近傍操作とルート間の近傍操作を用いた局所探索法を用いたアルゴリズムを提案した.
ルート内の近傍操作としては, 挿入近傍と交換近傍の2種類を提案し実装した.
挿入近傍とは, ルートのリクエストを1つ選択し他のルートに挿入する操作のことで, 交換近傍とは, ルートのリクエストを1つ選択し他のルートのリクエストと交換する操作のことである.
計算実験により2種類を比較したところ, 挿入近傍を用いたほうが交換近傍を用いるより良い解を見つけることができることがわかった.

時間枠及び乗車時間のペナルティを大きくすることで, ペナルティの値が小さい解を出力することが可能なので, 先行研究との解の精度の比較を行った. 規模の小さいインスタンスに対しては解の相違が20-30\%程度の解を得ることができたが, 規模の大きいインスタンスに関してはあまり良い結果を得ることができなかった. これは十分な試行回数を行うことができなかったことが理由と考えられる.

また, 本研究では目的関数をルートの長さとペナルティの変数の重み付き和としているため, 係数を変化させた際のルートの比較を行った. 計算結果より, 係数を変化させることでペナルティを優先的に小さくしたい場合や, 少しペナルティが大きくてもルートの長さも小さくしたい場合など, 様々な場合に対応するルートを出力することが可能であることが確認できた.

インスタンスで定められた車両数を1台減らした際の計算実験も行い, 比較を行った. 計算結果より, 車両数を1台減らした際には, 解の精度は少し悪くなり, ペナルティの値も大きくなるが, 実行可能なルートを出力できることがわかった. これにより, サービス自体のコストを下げながら運行を行うことが可能であると考えられる.



%%%%%%%%%%%%%%%%%%%%%%%%%%%%%%%%%%%%%%%%%%%%%%%%%%%%%%%
%修士論文の場合は英語のアブストも必要なので以下を記述して下さい
%学位の場合はいらないので, 以下は消して下さい.
\newpage
\begin{center}{\LARGE Local  search algorithm for Dial-a-ride problem\\ with convex time penalty}\\[0.5cm]
\end{center}
\hfill {\large 051400141\qquad Kiyoshi Takeda}\\[0.5cm]
\begin{center}
{\large \bf Abstract}\\
\end{center}
The Dial-a-Ride Problem (DARP) consists of designing vehicle routes and schedules for multiple users who specify pickup and delivery requests between origins and destination. The aim is to design a set of minimun cost vehicle route while accommodating all requests. Side constraints include vechile capacity, route duration, maximum ride time, etc. DARP arises in share taxi service or transporting people in health care service.

Dial-a-ride problem is a problem of determining visit order and assigning vehicles where we must satisfy the following constrains : (1) all registered vertices must be visited, (2) vehicles must start and terminate at depots, (3) pickup requests must be visited before their corresponding delivery request, (4) pairs of pickup and delivery requests must be served by the same vehicle.  Users can specify pickup and delivery location and earliest service time and latest service time. Users can also specify maximum ride time.

A number of papers about DARP has been published. most of the previous research deal with time windows and maximum ride time as hard constraints.

In this study, we consider with the DARP where time windows and maximum ride time are piecewise linear convex functions. By giving these constraints as soft constrains, users' dissatisfactions can be expressed flexibly. Therefore, Dial-a-ride problem with convex time penalty is a more generic Dial-a-ride problem.

We propose a local search algorithm with two operationson request vertices. One of the operations is intra-route operation that removes a request pair from a route and add the request pair in the same route. The other operations is inter-route operations and we made two types of algorithm to compare performance of inter-route operations.
One of the operations about inter-route is called relocate-inter, that removes a request pair from a route and add the request pair to another route. The other operation is called swap-inter, that removes a request pair from a route and exchange it with the request pair from another route.
We find that algorithm which used relocate-inter is better than algorithm which used swap-inter.

By making time window penalty large, we can compare the performance with the previous study. We confirm that we can find a route with about 30\% difference from previous study for small instances. On the other hand, we cannot find a good route for large instances. This may be due to insufficient number of iterations.

In this study, the objective function is a weighted sum of the length of the route and penalty and we perfom experiments with different coefficients.
We confirm that changing the penalty's coefficient result in a solution corresponding to various situations.

We reduced the number of vehicles by one and compare the result with the result with default number of vehicles. We confirm that we can find a feasible solution even if the number of vehicles is one less.
This result leads to reduced service costs such as fuel and time.




\thispagestyle{empty} 
\tableofcontents
\newpage
\setcounter{page}{1}
\pagestyle{plain}
\pagenumbering{arabic}



\chapter{はじめに}
これは卒論や修論のテンプレートです.
卒論や修論を作る前に,書き物に関する基本的な注意点を書いたページ
\footnote{\url{http://www.co.cm.is.nagoya-u.ac.jp/~yagiura/writing/writing.html}}
に書いたことをよく読んでください.
このページには加筆,訂正することがときどきあるので,資料を作るたびに見直してください.

発表のたびに配布資料を書く(あるいは前回の資料に加筆修正する)ことで,
論文のための文章を書くことや\LaTeX を使うことに慣れるとともに,
卒論・修論につながる文章を蓄積していってください.

このテンプレートはあくまでも参考ですので,
フォントサイズや行間等のスタイルを自由に変更してかまいません.

「はじめに」の節ではこの資料で何を書くのかを手短に説明してください.
どのような問題を対象にしてどのような手法を提案し,どのような結果が得られたのかを書くなどです.
また,研究背景についても関連研究の文献を挙げつつ述べましょう.
文献リストの書き方の例を挙げておきます.
\cite{GareyJohnson79}は本,
\cite{MHI13,YIG04}は論文誌の論文,
\cite{IYI05}は国際会議の予稿集の論文,
\cite{JohnsonMcGeoch97}は論文を集めた本やハンドブックに掲載された論文の例です.

必要な情報を
順序よく(つまり後ろを見ないと分からないことが出て来たりしないように),
明確に(曖昧さなく厳密に),
コンパクトに(冗長な表現を無くして手短に)書くよう心がけましょう.


構成は研究テーマや書きたいことによって様々ですが,
例えばある問題に対するアルゴリズムを提案して,
その効果を計算実験によって検証するようなテーマであれば,はじめにの節ののち,
問題の説明,提案手法の説明,計算実験の紹介,まとめという構成がしばしば用いられます.
以下の節ではそのような説明を書く際によく利用する書式や書くべきことの例をいくつか示しておきます.
\chapter{問題定義}\label{definition}
この章では,まずpickup and delivery problem(PDP)について説明し, そのあとに乗合タクシー問題(DARP)について詳しく説明する.
\section{Pickup and delivery problem}
pick and delivery problem (PDP)は, 荷物を出発地まで目的地まで輸送する問題である.PDPでは, 荷物を受け取る地点(pickup)と荷物を配送する地点(delivery)のペアからなる$n$個のリクエストと$m$台の車両が与えられたときに以下の制約を満たしつつ与えられたコストを最小化するルートを求める問題である.
\begin{enumerate}
 \item 与えられた$n$個のリクエスト全ての地点を訪問する.
 \item 全ての車両はデポから出発してデポに帰る.
 \item リクエストでペアになっている出発地と目的地は同じ車両が訪問する.
 \item それぞれのリクエストにおいて, 必ず出発地を訪問した後に目的地を訪問する.
\end{enumerate}
ここでのリクエストとは,荷物の輸送要求のことを指す.
与えられるコストとしては各頂点間の距離や時間などがある.よく制約として考えられるものは, リクエストの訪問点を決められた時間内に訪れなければならない時間枠制約や車両の容量を制限した容量制約などがある.また, 車両の種類が複数あり,車両によって最大容量が異なる多資源制約や,複数のデポを考慮する問題など,様々な拡張が考えられている.

\section{乗合タクシー問題(Dial-a-ride problem)}
乗合タクシー問題(Dial-a-ride problem, DARP)は, PDPを人の輸送に特化した問題である, DARPはPDPと違い人を輸送するため,車両に乗っている乗車時間が長いと利用者の不満がたまってしまう.そこで乗車時間やリクエストの訪問時間のずれなどで評価される不満度を考慮する必要がある.

顧客の満足度は, 利用者の乗車時間と出発地と目的地での待ち時間ではかる. 本研究では,利用者が乗車時刻, 降車時刻, 乗車時間のそれぞれに対して希望を持つことを考える. それぞれの希望はそれぞれ連続区分線形凸関数のペナルティ関数で表される. ペナルティ関数を区分線形関数にすることで, 任意の間隔ごとに関数を設定することが可能になり, 3分の遅延なら許容できるが20分の遅れは許容できないなどの表現が可能になる. また, 乗車時間に応じてペナルティをかけることができるので, 不満度を柔軟に表現することが可能になる. したがって, DARPをより汎用的に解くことが可能となる.

事前に全てのリクエストがわかっている問題を静的DARP, リクエストが全てはわかっておらず問題を解く過程でリクエストが次々と与えられる問題を動的DARPといい, 本研究では静的DARPを考える.

\section{乗合タクシー問題の問題複雑度}
DARPは, 使用する車両数を1, リクエストにおける出発地を全てデポとする巡回セールスマン問題(traveling salesman problem, TSP)ととらえることができる.TSPはNP困難\cite{TSP}であることが知られているため, DARPもNP困難である.

\chapter{定式化}\label{formulation}
乗合タクシー問題は, $G = (V, E)$の完全有向グラフ上で定義される. $V =\{\{0,2n+1\} \lor P \lor D\}$を頂点集合とし, $E = \{(i,j) \mid i,j\in V ,i \neq j\}\}$を各頂点間の辺集合とする.
ここで, デポを$0$と$2n+1$で表し, $V$の部分集合$P =\{1,\ldots,n \}$を乗車地点の集合, $V$の部分集合$D =\{n+1,\ldots,2n \}$を降車地点の集合とする.
$i \in P$, $i+n \in D$に対して, $V$上の2点のペア$(i,i+n)$は乗車地点の頂点$i$から目的地に対応する頂点$i+n$への乗客輸送リクエスト(以下, 単にリクエストと呼ぶ)とする.
それぞれの頂点$v_i \in V$には, 負荷$q_i$とサービス時間$s_i$が与えられる.

各リクエストは, $m$台の車両$k \in K = \{1,2,\ldots,m\}$で訪問される.
$\sigma$をルートの集合とし, $\sigma_k$を車両$k$の訪問する頂点の順列とすると, $\sigma = \{\sigma_1,\ldots,\sigma_m\}$である. また, $\sigma_k(h)$は車両$k$が$h$番目に訪問する頂点をあらわす.
頂点$i,j \in V$間の距離を$c_{ij}$, 時間を$t_{ij}$とする.
車両の容量を$Q$とし, 車両$k$の$\sigma _{k(h)}$における容量を$q_{\sigma_k(h)}$とする.
$x_{ij}$を0-1変数とし, 頂点$i,j$を結ぶ辺がルートに含まれれば1, そうでなければ0とする.
各車両は1つのデポから出発しデポに帰る. ここで, $n_k$は車両$k$が訪問するデポ以外の頂点の数である.

この問題に対して, 目的関数として
\begin{enumerate}
 \item 車両運用コスト最小化,
 \item 顧客満足度最大化
\end{enumerate}
の2つを考える. 車両運用コストとしては, 様々なコストが考えられるが本研究では車両が走行するルートの総距離とする. ルートの総距離を$d(\sigma)$とすると,
\begin{align*}
d(\sigma) = \sum_ {k\in K} \sum_{h=0}^{n_k} c_ {\sigma_{k(h)},\sigma_ {k(h+1)} }\\
\end{align*}
と表せる.
各リクエストにおける乗車時刻に対するペナルティ関数を$g^+_i$, 降車時刻に対するペナルティ関数を$g^-_i$, 乗車時刻に対するペナルティ関数を$g_i$とする.
頂点$i$でのサービス開始時刻を$\tau_i$とし, Iをリクエスト集合として, $I_\sigma$をルート$\sigma$に属するリクエストの集合とする.
また, 利用者の不満度を$t(\sigma)$とすると,
\begin{align*}
t(\sigma) = \sum_ {i \in I_\sigma} (g^+_i(\tau_i)+g^-_i(\tau_{i+n})+g_i(\tau_{i+n}-\tau_i))
\end{align*}
と表せる. 各ルートについて車両の割り当てとリクエストの訪問順が決まっている1つのルートが与えられた場合に, 各頂点でのサービス開始時刻を決定する必要がある. 本研究では, 目的関数と制約が全て線形の式で表すことが可能なので, 線形計画問題(linear programming problem, LP)として解くことができる.

本研究では, 2つの目的関数の重み付き和の最小化を考える. そうすることで, 少し顧客の不満度が上がっても良いのでルートの総距離を短くしたい場合や, 顧客の不満度を最優先で考えたい場合など, 様々な状況を考慮することができる.

それぞれの係数を定数$\alpha$と$\beta$としたとき, 以下のように定式化できる:
\begin{align*}
  &\textrm{minimize}   &&
  \alpha d(\sigma)+ \beta t(\sigma),\tag{1}\\
  &\textrm{subject to} && \sum_{i \in V} x_{ij} = 1 \ \ \ \ \ \ \ \ \ \ \ \ \ \ \ \ \ \ \ (j \in  P \cup D), \tag{2}\\
  &                    && \sum_{j \in V} x_{ij} = 1 \ \ \ \ \ \ \ \ \ \ \ \ \ \ \ \ \ \ \ (i \in  P \cup D), \tag{3}\\
  &                    && \sum_{k \in K} n_k = 2n,\tag{4}\\
  &                    && \sigma_k(0) = \sigma_k(n_k+1) = 0 \ \ \ \ \ \ (k \in K),\tag{5}\\
  &                    && 0 < q_{\sigma_k(h)} < Q\ \ \ \
  (h \in \{0,...,n_k\},k \in K),\tag{6}\\
  &                    && g_i(\tau_{i+n}-\tau_i) > s_i + t_{i,i+n}\ \ \ \ \ \ \ \ \ \ \ \ \ \  (i \in I),\tag{7}\\
  &                    && \tau_ {\sigma_k (h+1)} \geqq \tau_ {\sigma_k (h)} + s_{\sigma_k (h)} + t_ {\sigma_k (h),\sigma_k (h+1)}\\
  &                    &&  \ \ \ \ (h \in \{0,...,n_k\},k \in K).\tag{8}
\end{align*}
式$(2),(3)$は訪問点が1回ずつしか訪問されないことを, 式$(4)$は全てのリクエストが実行されることを表している. 式$(5)$は全ての車両がデポから出発してデポに帰ることを表している. 式$(6)$は車両の容量制約を表している. 式$(7),(8)$はリクエストの先行制約とサービス時刻に関する制約を表している.

\chapter{提案手法}\label{method}
本節では, 本研究で提案する手法について説明する. 本研究では, リクエストの割り当てと訪問順を局所探索法を用いて求める. それぞれの反復で得られたルートに対しては, LPソルバーを使って最適なサービス開始時刻を決定するアルゴリズムを提案する.
\section{初期解生成}
リクエストをランダムに選び, 車両$k$にリクエストのペアが連続となるようにルートの最後に挿入する. これを$k = 1 から m$まで繰り返した後, 未割り当てのリクエストがあれば$k = 1 $として同様の操作を続ける. これを未割り当てのリクエストがなくなるまで続ける.
このように生成することで、デポから出発してリクエスト全てを訪問し、同じ車両で出発地のあとに目的地を訪問するという制約を必ず守る初期解を生成することができる。

\section{制限の緩和}
本研究では, 局所探索を行う上でより自由に探索を行うために, 車両における容量制約を緩和し, 容量制約を破った時のペナルティを計算するペナルティ関数を定義する. 容量制約のペナルティ関数を目的関数に加えた評価関数を用いて解を評価することにより, 実行不可能解も探索可能になる.

車両の容量制約のペナルティは, 車両の最大容量を超えて乗った人数として, $QP$と表す. 車両$k$に対してルートの$i$番目を訪問後に容量を超えて乗っている人数を$QP_i^k$とすると,
\begin{align*}
  QP = \sum_{k \in K}\sum_{i \in n_k} QP_i^k,
\end{align*}
と表せる.
$\gamma$を定数とすると、ペナルティを加えた評価関数を$f(\sigma)$とすると
\begin{align*}
  f(\sigma) = \alpha d(\sigma)+ \beta t(\sigma) + \gamma QP,
\end{align*}
で定義する.

\section{局所探索法}
局所探索法(local search)とは、解を逐次的に改善させていく手法である。解に変化を少し加える操作を近傍操作と呼び、近傍操作によって生成されうる解の集合を近傍と呼ぶ。局所探索法では、適当な初期解からはじめ、現在の解の近傍内に、より良い解が存在すればその解に移動する、という操作を近傍内に改善がなくなるまで反復する方法である。本研究では、近傍操作は大きく分けてルート内の操作とルート間の操作の2種類を用いる。

\subsection{ルート内の近傍操作}
1つのルートの1つの頂点を選び、同じルート内の別の箇所に挿入し直す操作である。本研究では1つのルートで改善がなくなるまでルート内の近傍操作を行うが、探索する近傍サイズはルート内のデポを除いた頂点数を$a$とすると$a^2$とした。

\subsection{ルート間の近傍操作}
ルート間の近傍操作では、挿入近傍と交換近傍の2種類を実装し、計算結果を比較した。2種類ともに、ルート間の近傍操作を行い、現在の解の評価関数よりも改善した場合、操作後の近傍解に移動する。
\subsubsection{挿入近傍}
1つのルートから1つのリクエストペアを選び、別のルートに挿入する操作である。挿入する場所は評価関数が最も良くなる場所とする。
\subsubsection{交換近傍}
1つのルートから1つのリクエストペアを選び、別のルートのリクエストペアと交換する操作である。挿入近傍と同じく、新たに挿入する場所は、最も評価関数がよくなる場所とする。
\subsection{局所探索アルゴリズム}
時間枠及び乗車時間ペナルティ付き乗合タクシー問題に対する局所探索アルゴリズムとして、まずルート内で近傍操作を改善がなくなるまで行い、得られた局所最適解に対してルート間の近傍操作を行い、最良の箇所に挿入する。改善している場合、局所最適解を更新し、現在の解から移動する。

以下に、車両数を$m$、それぞれの車両$j$のルートにある頂点の数を$a_j$、$x_{init}$を初期解、$x$を現在解、$x_{init}$を暫定解とした際の提案手法の概要を示す。
\begin{algorithm}
 \caption{提案手法}
 \label{algo1}
 \begin{algorithmic}[1]%1を0にすると行番号なし.
  \STATE $x := x_{init}$とする。
  % \FOR{$j = 1$ to $m$}
  % \FOR{$k = 1$ to $a_j^2$}
  % \STATE ルート内の近傍操作を行う
  % \IF{改善した}
  % \STATE  $x_{best} := x$
  % \ENDIF
  % \ENDFOR
  % \ENDFOR
  \STATE 改善がなくなるまでルート内の近傍操作を行う
  \STATE ルート間の近傍操作を行う
  \STATE 変化があったルートに対してルート内の近傍操作を行う
  \IF{改善した}
  \STATE  $x_{best} := x$
  \ENDIF
  \STATE $x_{best}$を局所最適解として出力して終了.
 \end{algorithmic}
\end{algorithm}

\chapter{提案手法2}\label{method2}

\chapter{計算実験}\label{computational_result}
\section{実験環境}
実験に用いるプログラムはC++を用いて実装し、計算機はプロセッサ 1.4GHz Intel Core i5, メモリ 16GB 2133 MHz LPDDR3のmacOsを搭載したものを使用した。探索における最適なサービス開始時刻を決定する際のLPソルバーとしては、Gurobi Optimizer (ver 9.0.0)を使用した。

\section{問題例の作成方法}
DARPでは多くの既存研究があるが、本研究では時間枠及び乗車時間に対して区分線形で凸のペナルティ関数で与えている。このような問題設定のインスタンスは存在していないため、ベンチマークとしてよく使用されるCordeauらによって提供されている\cite{tabu}インスタンスに修正を加えて計算実験を行う。修正方法を以下に示す。

時間枠に関しては、サービス開始可能時刻を$e$、サービス開始最遅時間を$l$とすると、0以上$e$以下に対しては傾き-1、$e$から$l$に対しては値が0、$l$以上に対しては傾きが1となるような、区分数が3のペナルティ関数を作成する。この修正作業を全てのリクエストに対して行う。乗車時間に関しては、乗車時間の閾値を$L$とすると、0以上$L$以下に対しては値が0、$L$以上に対しては傾きが1となるような区分数2のペナルティ関数を作成する。この修正作業を全てのリクエストペアに対して行う。
\section{インスタンスについて}
計算実験に使用するインスタンスは、以下の特徴を持つ。
\begin{enumerate}
 \item 訪問点におけるサービス時間$d$を10とする。
 \item 乗降人数は1人とする。
 \item 訪問点$v_i$と$v_j$間の距離$c_{ij}$と時間$t_{ij}$は、2つの頂点のユークリッド距離とする。
 \item 乗車時間の閾値(最大乗車時間)を90とする。
 \item 車両の最大容量を6人とする。
 \item それぞれの車両のルートの最大の長さを480とする。
\end{enumerate}
以上の特徴を持つリクエスト数24から144のインスタンスを使用して計算実験を行った。
\section{実験結果}
ここでは、提案手法について行った計算実験の結果を示す。
顧客数と車両数がそれぞれ同じであるインスタンスが2つずつあるため、添字$a,b$で区別する。
\subsection{挿入近傍と交換近傍の比較}
\label{sec:insert}
局所探索におけるルート間の近傍操作として挿入近傍と交換近傍の比較を行った。
目的関数におけるペナルティ係数$\alpha,\beta$は1とする。
試行回数を$10^4$回としたときの計算結果を表\ref{insert}に示す。

ほぼ全てのインスタンスにおいて、挿入近傍による値のほうが交換近傍による値よりも良い性能であることが確認でき、平均で6.5\%良い性能であることが確認できた。

\subsection{既存研究との比較}
本研究での提案手法で、時間枠と乗車時間のペナルティの値を大きくすることで、既存研究との計算結果を比較することができる。\ref{sec:insert}で挿入近傍のほうが良い解を得られることが確認できたため、挿入近傍を用いて出力した解とCordeauらによって示された最良値を比較する。また、目的関数におけるペナルティを$\alpha=1,\beta=50$とする。
Cordeauらによって示された最良値は試行回数が$10^5$回の値であるため、同様に$10^5$回の試行を行い、結果を比較した。計算実験の値と最良解との相違を表\ref{cordeau}に示す。

\subsection{車両数を減らした計算結果}
サービスにおける


\begin{table}[]
\tabcolsep = 19pt
\renewcommand{\arraystretch}{0.8}
\caption{挿入近傍と交換近傍の比較}
\label{insert}
\begin{tabular}{ccccrcr}
\hline
    & \multicolumn{2}{c}{size} & \multicolumn{2}{c}{挿入近傍}                          & \multicolumn{2}{c}{交換近傍}                         \\ \cline{2-7}
問題例 & 顧客数           &  車両数          & 最良値                      & \multicolumn{1}{c}{計算時間} & 最良値                      & \multicolumn{1}{c}{計算時間} \\ \cline{2-7}
r1a & 24          & 3          & \multicolumn{1}{r}{216.005} & 2.14                     & \multicolumn{1}{r}{233.376} & 1.728                    \\
r2a & 48          & 5          & \multicolumn{1}{r}{514.503}  & 2.93                      & \multicolumn{1}{r}{566.30}  & 2.81                     \\
r3a & 72         & 7          & \multicolumn{1}{r}{960.476} & 4.917                     & \multicolumn{1}{r}{959.992} & 4.745                   \\
r1b & 24          & 3          & \multicolumn{1}{r}{195.6} & 2.417                     & \multicolumn{1}{r}{209.956} & 2.482                     \\
r2b & 48          & 5          & \multicolumn{1}{r}{437.568} & 3.340                     & \multicolumn{1}{r}{453.139} & 3.247                     \\
r3b & 72         & 7          & \multicolumn{1}{r}{921.741} & 4.658                   & \multicolumn{1}{r}{1031.662} & 4.719                 \\
& \multicolumn{1}{c}{} & \multicolumn{1}{l}{} &                          &                           &                         &                               \\
 \multicolumn{1}{c}{average} & \multicolumn{1}{c}{} &                          &   \multicolumn{1}{r}{540.982}                        &    \multicolumn{1}{r}{3.4}                      &  \multicolumn{1}{r}{575.738}  &     \multicolumn{1}{r}{3.288}                        \\
\hline
\end{tabular}
\end{table}



\begin{table}[]
\tabcolsep = 18pt
\renewcommand{\arraystretch}{0.8}
\caption{Best Known Scoreとの比較}
\label{cordeau}
\begin{tabular}{ccclllll}
\hline
        & \multicolumn{2}{c}{size}                    & \multicolumn{3}{c}{本研究}                                                        & \multicolumn{2}{c}{Besk Known Score}                       \\ \cline{2-8}
問題例     & n                    & m                    & \multicolumn{1}{c}{値} & \multicolumn{1}{c}{ペナルティ} & \multicolumn{1}{c}{GAP(\%)} & \multicolumn{1}{c}{1000} & \multicolumn{1}{c}{10000} \\ \cline{2-8}
r1a     & 24                   & 3                    & \multicolumn{1}{r}{212.49}    & \multicolumn{1}{r}{0.0}     & \multicolumn{1}{r}{11.8}   & \multicolumn{1}{r}{190.79}    & \multicolumn{1}{r}{190.02}     \\
r2a     & 48                   & 3                    & \multicolumn{1}{r}{386.03}    & \multicolumn{1}{r}{0.0}     & \multicolumn{1}{r}{27.7}   & \multicolumn{1}{r}{302.08}    & \multicolumn{1}{r}{302.08}     \\
r3a     & 72                   & 7                   & \multicolumn{1}{r}{763.27}    & \multicolumn{1}{r}{3.5}     & \multicolumn{1}{r}{43.4}   & \multicolumn{1}{r}{532.08}    & \multicolumn{1}{r}{532.08}     \\
r4a     & 120                   & 9                  & \multicolumn{1}{r}{1}    & \multicolumn{1}{r}{2}     & \multicolumn{1}{r}{3}   & \multicolumn{1}{r}{572.68}    & \multicolumn{1}{r}{572.78}     \\
r5a     & 144                  & 11                   & \multicolumn{1}{r}{1}    & \multicolumn{1}{r}{2}     & \multicolumn{1}{r}{3}   & \multicolumn{1}{r}{636.97}    & \multicolumn{1}{r}{636.97}     \\
r6a     & 24                   & 13                    & \multicolumn{1}{r}{1}    & \multicolumn{1}{r}{2}     & \multicolumn{1}{r}{3}   & \multicolumn{1}{r}{801.40}    & \multicolumn{1}{r}{801.40}     \\
r1b    & 48                  & 3                    & \multicolumn{1}{r}{195.6}    & \multicolumn{1}{r}{0.0}     & \multicolumn{1}{r}{18.9}   & \multicolumn{1}{r}{164.72}    & \multicolumn{1}{r}{164.46}     \\
r2b     & 72                  & 5                  & \multicolumn{1}{r}{388.43}    & \multicolumn{1}{r}{0.0}     & \multicolumn{1}{r}{31.0}   & \multicolumn{1}{r}{301.28}    & \multicolumn{1}{r}{296.06}     \\
r3b     & 24                   &7                   & \multicolumn{1}{r}{1}    & \multicolumn{1}{r}{2}     & \multicolumn{1}{r}{3}   & \multicolumn{1}{r}{498.20}    & \multicolumn{1}{r}{493.30}     \\
r4b     & 24                   & 9                   & \multicolumn{1}{r}{1}    & \multicolumn{1}{r}{2}     & \multicolumn{1}{r}{3}   & \multicolumn{1}{r}{548.89}    & \multicolumn{1}{r}{535.90}     \\
r5b     & 120                 & 11                   & \multicolumn{1}{r}{1}    & \multicolumn{1}{r}{2}     & \multicolumn{1}{r}{3}   & \multicolumn{1}{r}{592.65}    & \multicolumn{1}{r}{589.74}     \\
r6b     & 144               & 13                   & \multicolumn{1}{r}{1}    & \multicolumn{1}{r}{2}     & \multicolumn{1}{r}{3}   & \multicolumn{1}{r}{766.55}    & \multicolumn{1}{r}{743.60}     \\
        & \multicolumn{1}{l}{} & \multicolumn{1}{l}{} &                          &                           &                         &                          &                           \\
average & \multicolumn{1}{l}{} & \multicolumn{1}{l}{} &                          &                           &                         &       \multicolumn{1}{r}{524.91 }                    &       \multicolumn{1}{r}{515.38}                     \\ \hline
\end{tabular}
\end{table}

% \begin{table}[htbp]
%  \centering
%  \tabcolsep = 40pt
%  \renewcommand{\arraystretch}{0.8}
%  \caption{表の表示例}
%  \label{table12}
%  \begin{tabular}{lrr} \hline
%   問題例 & 挿入近傍 & 計算時間(秒)& 挿入近傍 & 計算時間(秒) \\ \hline
%   instance\_24\_3 &    123 & 10.1 \\
%   c10100 &    456 & 15.2 \\
%   c20100 &    789 & 20.3 \\ \hline
%  \end{tabular}
% \end{table}


% \begin{table*}
%  \centering
%  \tabcolsep = 19pt
%  \renewcommand{\arraystretch}{0.8}
%  \caption{2段組みスタイルにおいて幅の広い表を表示する例}
%  \label{table2}
%  \begin{tabular}{lrrcrr} \hline
%   &\multicolumn{2}{c}{既存手法} && \multicolumn{2}{c}{提案手法}\\ \cline{2-3} \cline{5-6}
%
%   問題例 & 最良値 & 計算時間(秒)&& 最良値 & 計算時間(秒) \\ \hline
%   c05100 &    123 &          10.1 &&    111 &          10.0 \\
%   c10100 &    456 &          15.2 &&    432 &          15.0 \\
%   c20100 &    789 &          20.3 &&    765 &          20.0 \\ \hline
%  \end{tabular}
% \end{table*}

\chapter{まとめと今後の研究計画}\label{conclution}
時間枠及び乗車時間ペナルティ付き乗合タクシー問題に対して、局所探索法による解法を提案した。近傍操作においては、挿入近傍と交換近傍の2種類を考え、それぞれを使用した場合の実験結果を比較した。挿入近傍を用いたほうが、交換近傍を用いるより良い精度の解を出力することが確認できた。
また、時間枠及び乗車時間のペナルティの値を大きくし、Cordeauらによる既存研究の研究結果と比較した。規模の小さいインスタンスに対しては相違30\%程度の解を出力したが、規模の大きいインスタンスに対してはあまり良い精度の解を出力することができなかった。
また、規模の小さいインスタンスに対して、車両数を1台減らした際の解を出力し、比較を行った。時間枠及び乗車時間のペナルティの値が発生してしまうものの、実行可能な解を出力できることがわかった。

今後の研究計画としては、局所最適解ではなく大域的な最適解を出力できるようにするため、局所探索ではない手法を提案していきたい。
また、規模の大きいインスタンスに対しても精度の良い解を得るために、近傍操作の見直しなどを行っていく。
また、本研究の提案手法では近傍操作によって得られたルートの訪問順の全てに対して線形計画問題をLPソルバーを使って解を出力したが、ある程度解の改善の可能性があるルートの訪問順に対してのみ線形計画問題を解くようにすることで計算時間を大幅に削減できると考える。したがって、解の改善可能性の判断についても手法の提案、実装を今後行っていきたい。

\chapter*{謝辞}
本研究の遂行にあたり, 熱心な指導と助言を頂きました柳浦睦憲教授, 胡艶楠助教に深く感謝の意を表します.
提案手法の検討やその有用性において活発に議論を頂きました, 高田氏, Vitor Mitsuo Hama氏に大変お世話になりました. 深くお礼申し上げます.
日々の研究室生活においては柳浦研究室の皆様にお世話になりました.
皆様のおかげで有意義な研究活動に勤しむことができました. 深くお礼申し上げます.





\addcontentsline{toc}{chapter}{参考文献}
\begin{thebibliography}{99}
 \bibitem{GareyJohnson79} M.R.~Garey and D.S.~Johnson,
	 {\it Computers and Intractability: A Guide to the Theory of NP-Completeness},
	 Freeman, New York, 1979.
 \bibitem{IYI05} S.~Imahori, M.~Yagiura and T.~Ibaraki,
	 Variable neighborhood search for the rectangle packing problem,
         {\it Proceedings of the 6th Metaheuristics International Conference} (MIC),
	 Vienna, Austria, August 22--26, 2005, pp.~532--537.
 \bibitem{JohnsonMcGeoch97} D.S. Johnson and L.A. McGeoch,
	 The traveling salesman problem: a case study,
	 in: E.H.L. Aarts and J.K. Lenstra (eds.),
	 {\it Local Search in Combinatorial Optimization},
	 John Wiley \& Sons, Chichester, 1997, pp.~215--310.
 \bibitem{MHI13} 真野洋平,橋本英樹,柳浦睦憲,
	 学生実験のスケジューリングシステムの構築,
	 オペレーションズ・リサーチ,58 (2013) 524--532.
 \bibitem{YIG04} M.~Yagiura, T.~Ibaraki and F.~Glover, 
	 An ejection chain approach for the generalized assignment problem,
	 {\it INFORMS Journal on Computing}, 16 (2004) 133--151.
\end{thebibliography}
\end{document}



% LocalWords:  ij Imahori Yagiura Ibaraki th Metaheuristics McGeoch Aarts lrr
% LocalWords:  Lenstra Chichester algorithmicx Fulkerson lrrcrr
%%presented by YUKI SAWAI 2015/12/22
%この卒論修論はhttp://www.co.cm.is.nagoya-u.ac.jp/~yagiura/writing/haifu_template/をもとに澤井佑樹が作っています
