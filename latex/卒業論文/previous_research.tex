\chapter{先行研究の紹介}\label{previous_research}
乗合タクシー問題に対して, 多くの研究がなされている. ここでは, DARPに関する既存研究をいくつか紹介する.

Jawらは, 各リスエストが時間枠を持ち, 複数の車両で容量制約, 乗車時間制約, 時間枠制約を守りつつ車両の移動コストを最小化する問題に対する近似解法として, 連続挿入法を提案した\cite{insertion}. この解法は, ルートに挿入した時の目的関数値の増加が最小になるようなリクエストを選択してルートに挿入していく解法である. この手法は大きく分けて, リクエストの挿入が実行可能となるルート内の挿入可能箇所を求めることと, 求めた挿入箇所の中でコストの増加が最も小さくなるような挿入箇所を選択することの2段階に分かれている.

Cordeauらは, DARPに対してタブー探索アルゴリズムを提案した\cite{tabu}. この手法では, あるルートからリクエストをひとつ取り除き,別のルートに挿入する, 挿入近傍を用いている. 解の探索では, 実行不可能な解の探索を許しており, 目的関数に制約違反ペナルティ関数が用意されている. 目的関数に制約違反ペナルティ関数を加えたものを$f(s)$とする. 解の評価関数に, リクエストとそのリクエストが属するルートのペアが解に採用された回数に比例するペナルティ関数を加える. このペナルティ関数を$p(s)$とすると, 解の評価関数は$f(s)とp(s)$の和で表すことができる. そうすることで, 頻繁に訪れられる解に訪問しないようになる.

Braekersらは, DARPに対しての発見的解法として焼きなまし法(simulated annealing, SA)を用いた手法を提案した\cite{SA}. この手法では, 挿入近傍, 交換近傍, 2-opt*, r-4-opt, 削除近傍の5つの近傍操作を使用している.
挿入近傍とは, ランダムにルートを選択して, そのルートからリクエストを1つ取り除いて別のルートの最適な位置に挿入する操作によって得られる解集合である.
交換近傍とは, ランダムにルートを選択して, そのルートからリクエストを1つ取り除き, 他のルートのリクエストと交換することによって得られる解集合である.
2-opt*とは, 異なるルート間の辺の交換操作のことである.  DARPでは, リクエストのペアは同じ車両が訪問しなければならないので, 乗客が乗っていない辺が交換の対象である.
r-4-optは, ルート内のリクエストの訪問順を変更する操作のことである.
上記4つの近傍操作は, 解を改善するために行われる操作であるが, 最後の削除近傍は, サービスにおいて使用される車両数を減らすことを目的とした操作である. 具体的には, ルートを1つランダム選択し, そのルートからリクエストを全て取り除く. そしてその取り除かれたリクエストをランダムな順番で別のルートに挿入していく操作のことである.

適応的巨大近傍探索 (adaptive large neighborhood search, ALNS) とは, 局所探索ある解の一部を 破壊して再構築する近傍探索のフレームワークである\cite{ALNS}. 破壊の発見的解法 (破壊方法) と再構築 の発見的解法 (構築方法) が複数用意され, 1 つの破壊方法と 1 つの構築方法を組み合わせて一つの近傍操作となる. 各反復ごとにどの破壊方法・構築方法を選ぶかは, 過去の反復での解の改善履歴を基に計算した値によって選択される. 破壊方法, 構築方法は問題に合わせて自由に決めることができる.
Ropkeらは, 時間枠付きのPDPに対してALNSを用いる手法を提案した\cite{pisinger}. Pisingerらは, 近傍となる破壊方法を7つ, 構築方法を2つ使用している.

可変近傍法(variable neighborhood search, VNS)とは, 複数の近傍を定義し, 暫定解に対してある近傍から1つ解を選択し, 局所探索を適用するという手順を繰り返す方法である. 局所探索の反復において, 暫定会の更新が起きたかによって次の解を選択する近傍を変えるのが特徴である. Dettiらは, DARPに対して可変近傍法(Variable neighborhood search,VNS)を用いた手法を提案した\cite{VNS}. この論文では, 交換近傍, 挿入近傍, 実行不可能解に対する挿入近傍, 連鎖近傍, 削除近傍の5つの近傍が使用されている.
