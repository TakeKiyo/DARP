\chapter{先行研究の紹介}\label{previous_research}
乗合タクシー問題に対して、多くの研究がなされている。ここでは、DARPに関する既存研究をいくつか紹介する。

Jawらは,各リスエストが時間枠を持ち、複数の車両で容量制約、乗車時間制約、時間枠制約を守りつつ車両の移動コストを最小化する問題に対する近似解法として、連続挿入法を提案した\cite{insertion}。この解法は、ルートに挿入した時の目的関数値の増加が最小になるようなリクエストを選択してルートに挿入していく解法である。この手法は大きく分けて、リクエストの挿入が実行可能となるルート内の挿入可能箇所を求めることと、求めた挿入箇所の中でコストの増加が最も小さくなるような挿入箇所を選択することの2段階に分かれている。

Cordeauらは,DARPに対してタブー探索アルゴリズムを提案した。この手法では、あるルートからリクエストをひとつ取り除き,別のルートに挿入する、挿入近傍を用いている。解の評価関数に、リクエストとそのリクエストが属するルートのペアが解に採用された回数に比例するペナルティ関数を加える。そうすることで、頻繁に訪れられる解に訪問しないようになる。\cite{tabu}.

Braekersらは、DARPに対しての発見的解法として焼きなまし法(Simulated Annealing,SA)を用いた手法を提案した\cite{SA}. この手法では、挿入近傍、交換近傍、2-opt*,r-4-opt、削除近傍の5つの近傍操作を使用している。

適応的巨大近傍探索 (Adaptive large neighborhood search, ALNS) とは, 局所探索ある解の一部を 破壊して再構築する近傍探索のフレームワークである\cite{ALNS}. 破壊の発見的解法 (破壊方法) と再構築 の発見的解法 (構築方法) が複数用意され, 1 つの破壊方法と 1 つの構築方法を組み合わせて一つ の近傍操作となる. 各反復ごとにどの破壊方法・構築方法を選ぶかは, 過去の反復での解の改善履 歴を基に計算した値によって選択される.
Pisingerらは、時間枠付きのPDPに対してALNSを用いる手法を提案した\cite{pisinger}。
