

%%%%%%%%%%%%%%%%%%%%%%%%%%%%%%%%%%%%%%%%%%%%%%%%%%%%%%%
%修士論文の場合は英語のアブストも必要なので以下を記述して下さい
%学位の場合はいらないので, 以下は消して下さい.
\newpage
\begin{center}{\LARGE Local  search algorithm for dial-a-ride problem\\ with convex time penalty}\\[0.5cm]
\end{center}
\hfill {\large 051400141\qquad Kiyoshi Takeda}\\[0.5cm]
\begin{center}
{\large \bf Abstract}\\
\end{center}
The dial-a-Ride Problem (DARP) consists of designing vehicle routes and schedules for multiple users who specify pickup and delivery requests between origins and destination. The aim is to design a set of minimun cost vehicle routes while accommodating all requests. Side constraints include vechile capacity, route duration, maximum ride time, etc. The DARP arises in share taxi service or transporting people in health care service.

dial-a-ride problem is a problem of determining visit order and assigning vehicles where we must satisfy the following constrains : (1) all registered vertices must be visited, (2) vehicles must start and terminate at depots, (3) pickup requests must be visited before their corresponding delivery request, (4) pairs of pickup and delivery requests must be served by the same vehicle.  Users can specify pickup and delivery location and earliest service time and latest service time. Users can also specify maximum ride time.

A number of papers about DARP has been published. most of the previous research deal with time windows and maximum ride time as hard constraints.

In this study, we consider with the DARP in which time windows and maximum ride time are piecewise linear convex functions. By giving these constraints as soft constrains, users' dissatisfactions can be expressed flexibly. Therefore, dial-a-ride problem with convex time penalty is a more generic dial-a-ride problem.

We propose a local search algorithm with two operations on request vertices. One of the operations is intra-route operation that removes a request pair from a route and add the request pair to the same route. The other operation is inter-route operation and we made two types of algorithm to compare performance of inter-route operations.
One of the operations about inter-route is called relocate-inter, that removes a request pair from a route and add the request pair to another route. The other operation is called swap-inter, that removes a request pair from a route and exchange it with the request pair from another route.
We find that algorithm which used relocate-inter is better than algorithm which used swap-inter.

By making time window penalty large, we can compare the performance with the previous study. We confirm that we can find a solution with about 30\% difference from previous study for small instances. On the other hand, we cannot find a good solution for large instances. This may be due to insufficient number of iterations.

In this study, the objective function is a weighted sum of the length of the routes and penalty and we perfom experiments with different coefficients.
We confirm that changing the penalty's coefficient result in a solution corresponding to various situations.

We reduced the number of vehicles by one and compare the result with the result with default number of vehicles. We confirm that we can find a feasible solution even if the number of vehicles is one less.
This result leads to reduced service costs such as fuel and time.
