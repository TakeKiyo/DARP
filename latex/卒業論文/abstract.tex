

%%%%%%%%%%%%%%%%%%%%%%%%%%%%%%%%%%%%%%%%%%%%%%%%%%%%%%%
%修士論文の場合は英語のアブストも必要なので以下を記述して下さい
%学位の場合はいらないので, 以下は消して下さい.
\newpage
\begin{center}{\LARGE Local  search algorithms for the dial-a-ride problem\\ with convex time penalty}\\[0.5cm]
\end{center}
\hfill {\large 051400141\qquad Kiyoshi Takeda}\\[0.5cm]
\begin{center}
{\large \bf Abstract}\\
\end{center}
The dial-a-ride problem (DARP) consists of designing vehicle routes and schedules for multiple users who specify pickup and delivery requests between origins and destinations. The aim is to design a set of minimun cost vehicle routes while accommodating all requests. Side constraints include a vechile capacity, the route duration, the maximum ride time, etc. The DARP arises in sharing taxi service or transporting people in health care service.

The dial-a-ride problem is a problem of determining a visiting order and assigning vehicles where we must satisfy the following constrains : (1) all registered vertices must be visited, (2) vehicles must start and terminate at depots, (3) pickup requests must be visited before their corresponding delivery request, (4) pairs of pickup and delivery requests must be served by the same vehicle.  Users can specify pickup and delivery location, the earliest service time, and the latest service time. Users can also specify the maximum ride time.

A number of papers about DARP have been published. Most of the previous research deal with time windows and the maximum ride time as hard constraints.

In this study, we consider with the DARP in which time windows and the maximum ride time are piecewise linear convex functions. By giving these constraints as soft constrains, users' dissatisfactions can be expressed flexibly. Therefore, the dial-a-ride problem with convex time penalty is a more generic dial-a-ride problem.

We propose a local search algorithm on request vertices. We made three types of algorithm to compare performance of inter-route operations.
One of the operations is called relocate-inter that removes a request pair from a route and add the request pair to another route. Another operation is called swap-inter that removes a request pair from a route and exchange it with the request pair from another route. The other operation is a mixture of two operations that uses relocate-inter and swap-inter alternately.
We find that the algorithm which used relocate-inter and the algorithm which used mixture of two operations are better.

By making the time window penalty large, we can compare the performance with the previous study. We confirm that we can find a solution with about 40\% difference from previous study for all instances. 
In this study, the objective function is a weighted sum of the length of the routes and penalty and we perform experiments with different coefficients.
We confirm that changing the penalty's coefficient result in a solution corresponding to various situations.

We reduced the number of vehicles by one and compare the result with the one with default number of vehicles. We confirm that we can find a feasible solution even if the number of vehicles is one less.
This result leads to reduced service costs such as fuel and time.
