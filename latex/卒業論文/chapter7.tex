\chapter{まとめと今後の研究計画}\label{conclution}
時間枠及び乗車時間ペナルティ付き乗合タクシー問題に対して、局所探索法による解法を提案した。近傍操作においては、挿入近傍と交換近傍の2種類を考え、それぞれを使用した場合の実験結果を比較した。挿入近傍を用いたほうが、交換近傍を用いるより良い精度の解を出力することが確認できた。
また、時間枠及び乗車時間のペナルティの値を大きくし、Cordeauらによる既存研究の研究結果と比較した。規模の小さいインスタンスに対しては相違30\%程度の解を出力したが、規模の大きいインスタンスに対してはあまり良い精度の解を出力することができなかった。
また、規模の小さいインスタンスに対して、車両数を1台減らした際の解を出力し、比較を行った。時間枠及び乗車時間のペナルティの値が発生してしまうものの、実行可能な解を出力できることがわかった。

今後の研究計画としては、局所最適解ではなく大域的な最適解を出力できるようにするため、局所探索ではない手法を提案する。
また、規模の大きいインスタンスに対しても精度の良い解を得るために、近傍操作の見直しなどを行う。
また、本研究の提案手法では近傍操作によって得られたルートの訪問順の全てに対して線形計画問題をLPソルバーを使って解を出力したが、ある程度解の改善の可能性があるルートの訪問順に対してのみ線形計画問題を解くようにすることで計算時間を大幅に削減できると考える。したがって、解の改善可能性の判断についても手法の提案、実装を今後行っていく。
