\chapter{はじめに}
近年,新しい移動手段として乗合タクシーサービスの需要が増加してきている.また,老人や介護が必要な人を自宅などからヘルスケアセンターなどまで輸送するようなサービスも,高齢化に伴い需要が増加してきている.
都心部では、乗合タクシーの実証実験がさかんに行われたりしており、実用化が進んできている.
これらのサービスにおいて、利用者の満足度を考慮しつつサービスの実行にかかるコストを最小化する問題は乗合タクシー問題(Dial-a-Ride problem, DARP)と呼ばれる.

DARPでは、利用者を出発地から目的地まで輸送するサービスにおいて、リクエスト全ての点を訪問すること、車両はデポから出発しデポに戻ること、リクエストのペアは必ず同じ車両が訪問すること、出発地より後に到着点を訪問することの全てを満たすルートと車両の割り当てを考える問題である。利用者は、出発地と目的地の場所、またその地点に車両が到着してほしい時間枠を指定することができる。
また、輸送の時間に対して、最大乗車時間を指定することができる。
乗合タクシー問題に対しては,多くの研究がなされている.多くの先行研究では,利用者の最大乗車時間とリクエストの乗車時刻と降車時刻に対しての時間枠をハード制約として与えている.

本研究では,時間枠及び乗車時間をペナルティ関数で与えてソフト制約とする乗合タクシー問題について考える。ペナルティ関数は区分線形凸関数であるとする。区分線形関数にすることで,時間枠より少しの遅延は許容できる場合など,様々なケースを柔軟に考慮することができるようになり,先行研究より汎用的な問題とすることができる。提案手法としては、挿入近傍と交換近傍を用いた局所探索法を提案する。挿入近傍とは、ルートのリクエストを1つ選択し、他のルートに挿入する操作のことで、交換近傍とは、ルートのリクエストを1つ選択し、他のルートのリクエストと交換する操作のことである。

3章ではpickup and delivery problem(PDP)を紹介し,乗合タクシー問題(DARP)について詳しく説明する.
