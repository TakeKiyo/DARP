\chapter{はじめに}
近年, 新しい移動手段として乗合タクシーサービスの需要が増加してきている. また,老人や介護が必要な人を自宅などからヘルスケアセンターなどまで輸送するようなサービスも, 高齢化に伴い需要が増加してきている.
都心部では, 乗合タクシーの実証実験がさかんに行われており, 実用化が進んできている. これらのサービスでは, サービス提供者はなるべく多くの顧客のリクエストを実行しつつ, サービスにかかるコスト(運行時間やルートの総距離)を少なくすることが重要となる. 一方で利用者は, 希望通りの時間にサービスを利用して, なるべく短い時間で出発地から目的地に着くことが重要である. つまり, サービス提供者と利用者が互いに満足のいくルートを作成することが必要である.

乗合タクシーサービスにおいて, 利用者の満足度を考慮しつつサービスの実行にかかるコストを最小化する問題は乗合タクシー問題(dial-a-ride problem, DARP)と呼ばれる.
乗合タクシー問題の制約としては, 車両の乗車人数の上限(容量制約), 車両が回るルートの長さの上限(最大ルート距離), 顧客が車両に乗っている時間の長さの上限(最大乗車時間)が与えられる.

乗合タクシー問題に対しては, 多くの先行研究があり, 様々な手法が提案されている. 先行研究の一部を2章で詳しく紹介する. また, 複数のデポや異種車両などのさまざまな制約を考慮する研究もある. しかし先行研究では, 乗降時刻を何時から何時までという時間枠で定めている. また, 乗車時間に関しても何分までなら許容できるという最大乗車時間で定めていることが一般的である.

本研究では, 制約として時間枠及び乗車時間に対するペナルティ関数を与えてソフト制約とする乗合タクシー問題について考える. ペナルティ関数は区分線形凸関数であるとする. 区分線形関数にすることで, ルート全体の効率を上げるために多少の遅延を許容できる場合など, 様々なケースを柔軟に考慮することができるようになる. また, 乗車時間の増加につれてペナルティの値も大きくなるので, 顧客の不満度をより柔軟に表現することが可能となる. これらの点より, 時間枠及び乗車時間ペナルティ付き乗合タクシー問題は先行研究より汎用的な問題であると言える.

3章ではpickup and delivery problem (PDP)を紹介し, 乗合タクシー問題(DARP)について詳しく説明する. 4章では時間枠及び乗車時間ペナルティ付き乗合タクシー問題の定式化を行う. 5章では本問題に対する提案手法を紹介する.
6章では, 挿入近傍と交換近傍と2種類を交互に行う操作による解の精度の比較と既存研究の最良解との比較を行った結果を紹介する. また, 目的関数を2つの変数の重み付き和としているため, 係数を変化させた際の計算結果の違いについても考察を行う. そして, 車両数を減らした際の結果も紹介する.
